\documentclass[journal,12pt,twocolumn]{IEEEtran}

\usepackage{setspace}
\usepackage{gensymb}
\singlespacing
\usepackage[cmex10]{amsmath}

\usepackage{amsthm}

\usepackage{mathrsfs}
\usepackage{txfonts}
\usepackage{stfloats}
\usepackage{bm}
\usepackage{cite}
\usepackage{cases}
\usepackage{subfig}

\usepackage{longtable}
\usepackage{multirow}
\usepackage{scalerel}
\usepackage{enumitem}
\usepackage{mathtools}
\usepackage{steinmetz}
\usepackage{tikz}
\usepackage{circuitikz}
\usepackage{verbatim}
\usepackage{tfrupee}
\usepackage[breaklinks=true]{hyperref}
\usepackage{graphicx}
\usepackage{tkz-euclide}
\usepackage{scalerel}

\usetikzlibrary{calc,math}
\usepackage{listings}
    \usepackage{color}                                            %%
    \usepackage{array}                                            %%
    \usepackage{longtable}                                        %%
    \usepackage{calc}                                             %%
    \usepackage{multirow}                                         %%
    \usepackage{hhline}                                           %%
    \usepackage{ifthen}                                           %%
    \usepackage{lscape}     
\usepackage{multicol}
\usepackage{chngcntr}

\DeclareMathOperator*{\Res}{Res}

\renewcommand\thesection{\arabic{section}}
\renewcommand\thesubsection{\thesection.\arabic{subsection}}
\renewcommand\thesubsubsection{\thesubsection.\arabic{subsubsection}}

\renewcommand\thesectiondis{\arabic{section}}
\renewcommand\thesubsectiondis{\thesectiondis.\arabic{subsection}}
\renewcommand\thesubsubsectiondis{\thesubsectiondis.\arabic{subsubsection}}


\hyphenation{op-tical net-works semi-conduc-tor}
\def\inputGnumericTable{}                                 %%

\lstset{
%language=C,
frame=single, 
breaklines=true,
columns=fullflexible
}
\begin{document}

\newcommand\givenbase[1][]{\:#1\lvert\:}
\newcommand{\BEQA}{\begin{eqnarray}}
\newcommand{\EEQA}{\end{eqnarray}}
\newcommand{\define}{\stackrel{\triangle}{=}}
\bibliographystyle{IEEEtran}
\raggedbottom
\setlength{\parindent}{0pt}
\providecommand{\mbf}{\mathbf}
\providecommand{\pr}[1]{\ensuremath{\Pr\left(#1\right)}}
\providecommand{\qfunc}[1]{\ensuremath{Q\left(#1\right)}}
\providecommand{\sbrak}[1]{\ensuremath{{}\left[#1\right]}}
\providecommand{\lsbrak}[1]{\ensuremath{{}\left[#1\right.}}
\providecommand{\rsbrak}[1]{\ensuremath{{}\left.#1\right]}}
\providecommand{\brak}[1]{\ensuremath{\left(#1\right)}}
\providecommand{\lbrak}[1]{\ensuremath{\left(#1\right.}}
\providecommand{\rbrak}[1]{\ensuremath{\left.#1\right)}}
\providecommand{\cbrak}[1]{\ensuremath{\left\{#1\right\}}}
\providecommand{\lcbrak}[1]{\ensuremath{\left\{#1\right.}}
\providecommand{\rcbrak}[1]{\ensuremath{\left.#1\right\}}}
\theoremstyle{remark}
\newtheorem{rem}{Remark}
\newcommand{\sgn}{\mathop{\mathrm{sgn}}}
\providecommand{\abs}[1]{\vert#1\vert}
\providecommand{\res}[1]{\Res\displaylimits_{#1}} 
\providecommand{\norm}[1]{\lVert#1\rVert}
%\providecommand{\norm}[1]{\lVert#1\rVert}
\providecommand{\mtx}[1]{\mathbf{#1}}
\providecommand{\mean}[1]{E[ #1 ]}
\providecommand{\fourier}{\overset{\mathcal{F}}{ \rightleftharpoons}}
%\providecommand{\hilbert}{\overset{\mathcal{H}}{ \rightleftharpoons}}
\providecommand{\system}{\overset{\mathcal{H}}{ \longleftrightarrow}}
	%\newcommand{\solution}[2]{\textbf{Solution:}{#1}}
\newcommand{\solution}{\noindent \textbf{Solution: }}
\newcommand{\cosec}{\,\text{cosec}\,}
\providecommand{\dec}[2]{\ensuremath{\overset{#1}{\underset{#2}{\gtrless}}}}
\newcommand{\myvec}[1]{\ensuremath{\begin{pmatrix}#1\end{pmatrix}}}
\newcommand{\mydet}[1]{\ensuremath{\begin{vmatrix}#1\end{vmatrix}}}
\numberwithin{equation}{subsection}
\makeatletter
\@addtoreset{figure}{problem}
\makeatother
\let\StandardTheFigure\thefigure
\let\vec\mathbf
\renewcommand{\thefigure}{\theproblem}
\def\putbox#1#2#3{\makebox[0in][l]{\makebox[#1][l]{}\raisebox{\baselineskip}[0in][0in]{\raisebox{#2}[0in][0in]{#3}}}}
     \def\rightbox#1{\makebox[0in][r]{#1}}
     \def\centbox#1{\makebox[0in]{#1}}
     \def\topbox#1{\raisebox{-\baselineskip}[0in][0in]{#1}}
     \def\midbox#1{\raisebox{-0.5\baselineskip}[0in][0in]{#1}}
\vspace{3cm}
\title{AI 1103 - Assignment 8}
\author{T. Rohan \\ CS20BTECH11064}
\maketitle
\newpage
\bigskip
\renewcommand{\thefigure}{\theenumi}
\renewcommand{\thetable}{\theenumi}
Download latex codes from 
\begin{lstlisting}
https://github.com/rohanthota/Assignment_8/Assignment 8.tex
\end{lstlisting}
\section*{\emph{Question}}
Suppose n units are drawn from a population of N units sequentially as follows. A random sample
\begin{align}
    U_1, U_2, ... U_N \text{ of size N, drawn from }U\brak{0, 1} 
\end{align} 
The k-th population unit is selected if 
\begin{align}
    U_k<\frac{n - n_k}{N-k+1}, k = 1, 2, ..N. \text{where, } n_1=0, n_k = 
\end{align}
number of units selected out of first k-1 units for each k = 2, 3, ..N. Then,
\begin{enumerate}
    \item The probability of inclusion of the second unit in the sample
    \begin{align}
        \text{ is } \frac{n}{N}
    \end{align}
    \item The probability of inclusion of the first and the second unit in the sample
    \begin{align}
        \text{ is } \frac{n \brak{n-1}}{N \brak{N-1}}
    \end{align}
    \item The probability of not including the first and including the second unit in the sample
    \begin{align}
        \text{ is } \frac{n \brak{N-n}}{N \brak{N-1}}
    \end{align}
    \item The probability of including the first and not including the second unit in the sample
    \begin{align}
        \text{ is } \frac{n \brak{n-1}}{N \brak{N-1}}
    \end{align}
\end{enumerate}
\section*{\emph{Solution}}
\begin{align}
\text{Defining random variable }X\in\cbrak{0, 1, 2, ..N}
\\\text{Where, } X = i \text{ when ith unit is included.}
\end{align}

The first unit in the sample is included if
\begin{align}
    U_1 < \frac{n-n_1}{N-1+1}  
    \\\text{Here, } n_1=0 \text{ is given in the qn.}
    \\\therefore \pr{X=1} = \frac{n}{N}
\end{align}
\begin{enumerate}
    \item For k=2, 
\begin{align}
    n_2 = 1 \text{ when, first unit is included.}
\end{align}
\begin{align}
    U_2 < \frac{n-n_2}{N-2+1} \brak{ = \frac{n-1}{N-1}}
\end{align}
\begin{align}
    \therefore \pr{X=2 \mid X=1} &= \frac{n-1}{N-1}
\end{align}
\pr{X=1, X=2} 
\begin{align}
     = \pr{X=2 \mid X=1} &\times \pr{X=1}
    \\\therefore \pr{X=1, X=2} &= \frac{n \brak{n-1}}{N \brak{N-1}}
\end{align}
\begin{align}
    n_2 = 0 \text{ when, first unit is not included.}
    \\U_2 < \frac{n-n_2}{N-2+1} \brak{= \frac{n}{N-1}}
    \\\therefore \pr{X=2 \mid X\neq 1} = \frac{n}{N-1}
\end{align}
\pr{X\neq1, X=2}
\begin{align}
     = \pr{X=2 \mid X\neq1} \times \pr{X\neq1}
\end{align}
\begin{align}
    \therefore \pr{X\neq1,X=2} = \brak{1-\frac{n}{N}}\times\frac{n}{N-1}
\end{align}
\begin{align}
    \therefore \pr{X\neq1,X=2} = \frac{n \brak{N-n}}{N \brak{N-1}}
\end{align}
From \brak{0.0.16} and \brak{0.0.22}
\begin{align}
    \pr{X=2} = \frac{n \brak{n-1}}{N \brak{N-1}} + \frac{n \brak{N-n}}{N \brak{N-1}} = \frac{n}{N}
\end{align}
Hence, option 1 is correct.
\item From \brak{0.0.16} 
\begin{align}
    \pr{X=1, X=2} = \frac{n \brak{n-1}}{N \brak{N-1}}
\end{align}
Hence, option 2 is correct.
\item From \brak{0.0.22} 
\begin{align}
    \pr{X\neq1,X=2} = \frac{n \brak{N-n}}{N \brak{N-1}}
\end{align}
Hence, option 3 is correct.
\item 
\begin{align}
\pr{X=1, X\neq2} = \frac{n}{N} \times \brak{1-\frac{n}{N}} = \frac{n \brak{N-n}}{N^2}
\end{align}
Hence, option 4 is incorrect.

Therefore, Options 1, 2, 3 are correct
\end{enumerate}
\end{document}






















